\documentclass[dutch,course]{lecture}
\usepackage[utf8]{inputenc}

\title{KUL QM 2021}
\author{Willem Vanhulle}
\date{11}{02}{2021}
\dateend{30}{06}{2021}
%\date{February 2021}

\begin{document}

\maketitle


\lecture[theorie]{10}{02}{2021}
\section*{Inleiding door prof}

\subsection*{Praktisch}
Handboek: Griffiths.

\begin{itemize}

\item
  Minder gedetailleerd
\item
  Meer to-the-point
\end{itemize}

Oefeningen:

\begin{itemize}

\item
  Xuao Zhang: vragen of opmerkingen (mogelijks kort)
\item
  Vanaf vrijdag
\end{itemize}

\subsection*{Inhoud vak}

Leerstof:

\begin{itemize}

\item
  Al gekend, maar iets wiskundiger, vandaag gespecificeerd:
  \begin{itemize}
\item
  \textbf{Hoofdstuk 1} : golffunctie
\item
  \textbf{Hoofdstuk 2} : tijdsonafhankelijke schrodingervgl.
  \end{itemize}
\item
  Nieuwe leerstof, een hoofdstuk per les:
  \begin{itemize}
\item
  Appendix over linear algebra (huiswerk)
\item
  Hoofdstuk 3: formalisme
\item
  Hoofdstuk 4: QM in 3d (volledig nieuw)
\item
  Hoofdstuk 5: identieke deeltjes, gedeeltelijk gezien in les

  \begin{itemize}
  
  \item
    Vastestof fysica in andere cursus
  \item
    Statistische aspect ook in andere cursus
  \end{itemize}
\item
  Hoofdstuk 6: tijds-onafhankelijke stooringsrekening
\item
  Hoofdstuk 7: variational principle, compuationele methode om problemen
  op te lossen
\item
  Hoofdstuk 8: WKB, alleen indien tijd over
\item
  Hoofdstuk 9: wel heel belangrijk, tijdsafhankelijke storingsrekening
\item
  Hoofdstuk 11: scattering
  \end{itemize}
\end{itemize}


\subsection*{Examen}

Examen:

\begin{itemize}

\item
  Schriftelijk
\item
  Fysiek
\item
  Formularium (niet gelijk aan oude formularium bij Bransden)
\item
  Gesloten boek, geen boek
\item ongeveer vier vragen: 2 theorie, 2 oefeningen
\end{itemize}

\section{The Wave equation}
Hoofdstuk 1

\subsection*{Schroedinger vergelijking}

Al gekend

\subsection*{Statistische interpretatie}

Toestand deeltjes zit volledig vervat in golffunctie


Volgens de Max-Born interpretatie is de absolute waarde van de golffunctie proportioneel met de waarschijnlijkheid.

Hoe kan de uitslag van een meting voorspeld worden? Denk bijvoorbeeld aan de  positie van een deeltje.

We veronderstellen dat een meting mogelijk is. Wat is het resultaat?

\begin{itemize}

\item
  niet in staat om informatie te geven over waar een deeltje zich bevond
  voor een meting werd uitgevoerd.
\item
  Alleen een waarschijnlijkheidsverdeling van positie voor meting is
  gekend.
\item
  Na meeting ziet waarschijnlijksverdeling er anders uit.
\end{itemize}

Verborgen variabelen worden toegevoegd in Pilot waves van Bohm, maar dat
is eigenlijk niet nodig.

Bijkomende informatie is er niet in QM.

Na meting verandert de golffunctie

Einstein die de pionier is van de Relativiteitstheorie:

\begin{itemize}

\item
  Vond QM wandeling op ijs
\item
  Heeft meeste bijgedragen aan concepten
\end{itemize}

Herhaling van Statistiek en rest hoofdstuk 1 is niet besproken


\section{Time-independent Schr\"odinger equation}

\begin{itemize}

\item
  Hoofdstuk 2.1-2.2: niet besproken
\item
  Hoofdstuk 2.3.
  \begin{itemize}
\item
  Algebraische methode (komt aan bod tijdens oefeningen)
\item
  Analytische method (komt niet aan bod, te wiskundig)
  \end{itemize}
\item
  Vrij deeltje
  \begin{itemize}
\item
  Wave packet (zie oefeningen)
\item
  Dispersie relation (zie oefeningen)
\end{itemize}
\item
  Delta-function potential (2.5)
\item
  Eindig diepe put potentiaal (2.6)
  \begin{itemize}
\item
  Iets meer detail
\item
  Zelfstudie
\item
  Veel rekenwerk
\item
  Heeft andere situaties als limietgeval
  \end{itemize}
\item

  Dubbele vierkante put (komt later aan bod)
  \begin{itemize}
\item
  Model voor storingsrekening,
\item
  Kan oscillaties molecule met meerdere atomen beschrijven
  \end{itemize}
\end{itemize}

Volgende les hoofdstuk 3.








\lecture[oefeningen]{12}{02}{2021}



% \file{See recording on Toledo}

\subsection*{Harmonic oscillator}
\subsection*{The Free particle}
\subsection*{Delta-function potential}

2.5 in Griffiths.



\section{Formalism}

\lecture[theorie]{17}{02}{2021}


\lecture[oefeningen]{19}{02}{2021}


Assignment will be discussed by Indeku and Zhang

\begin{problem}
Probleem
\end{problem}

\end{document}
